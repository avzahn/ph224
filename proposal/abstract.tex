\documentclass[12pt]{article}
\usepackage[utf8]{inputenc}
\usepackage[margin=0.6in]{geometry}

\usepackage{cite}

\begin{document}

\newcommand{\araa}{Annual Review of Astron and Astrophys}
\newcommand{\aap}{Astronomy and Astrophysics}
\newcommand{\apj}{Astrophysical Journal}

\section*{abstract}

Polarized thermal emission from insterstellar dust has recently and unexpectedly been discovered to be a limiting foreground for CMB gravitational wave searches. Foreground subtraction for future instruments, and the future of the field with it, depends on our understanding of the dust spectral energy distribution in the millimeter range. The LMT's SPEED camera's ability to simultaneously measure 2.1, 1.3, 1.1, and .85 mm intensity with a large collecting area provides an opportunity to improve dust SED models over sightlines already mapped by CMB instruments. We propose to deeply integrate a .25 square degree field with LMT SPEED near RA 0\(^\circ\) Dec -57.5\(^\circ\). This corresponds to a single pixel in the BICEP2/Keck Array 150 GHz map, providing spectral overlap with both BICEP2/Keck at 150 GHz and Planck maps at 353 GHz while covering spectral gaps at 232 GHz and 273 GHz.


\section*{choice of references}

\cite{bicep_planck} and \cite{planck_dust} contain dedicate sections to reviewing the state of the CMB foreground problem from a cosmology perspective.

\cite{andersson} reviews the state of astrophysical theory on the polarized dust SED. The takeaway is that a lot of progress has been made near the one micron emission peak, but that we don't have a lot of predictive power in the millimeter tails of the emission where the CMB is.

\cite{compiegne} and \cite{draine_li} describe different approaches to constraining the dust population properties given SED measurements. I think it's worth seeing if the proposed measurements can do useful dust physics.


\bibliography{bibfile}
\bibliographystyle{plain}

\end{document}