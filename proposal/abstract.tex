\documentclass[12pt]{article}
\usepackage[utf8]{inputenc}
\usepackage[margin=0.6in]{geometry}

\usepackage{cite}

\begin{document}

\newcommand{\araa}{Annual Review of Astron and Astrophys}
\newcommand{\aap}{Astronomy and Astrophysics}

\section*{abstract}

Polarized thermal emission from insterstellar dust has recently and unexpectedly been discovered to be a limiting foreground for CMB gravitational wave searches. Foreground subtraction for future instruments, and the future of the field with it, depends on our understanding of the dust spectral energy distribution in the millimeter range. The LMT's SPEED camera's ability to simultaneously measure 2.1, 1.3, 1.1, and .85 mm intensity with a large collecting area provides an opportunity to improve dust SED models over sightlines already mapped by CMB instruments. We propose to deeply integrate a .25 square degree field with LMT SPEED near RA 0\(^\circ\) Dec -57.5\(^\circ\). This corresponds to a single pixel in the BICEP2/Keck Array 150 GHz map, and fills gaps

\section*{choice of references}

\cite{andersson}

\bibliography{bibfile}
\bibliographystyle{plain}

\end{document}